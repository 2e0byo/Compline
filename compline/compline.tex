\documentclass[a5paper,openany,10pt]{memoir}
% Packages
\usepackage{fontspec}
\setmainfont{TeX Gyre Pagella} %palatino clone
\usepackage[yyyymmdd,hhmmss]{datetime}
% \usepackage{microtype}
\usepackage{etoolbox}
\usepackage{lettrine}
\usepackage[british, latin]{babel}
\usepackage{paracol}
\usepackage{scalerel}
\usepackage{verse,anyfontsize}
\usepackage{stackengine}
\usepackage[hidelinks, final]{hyperref}
\usepackage{multicol}
\usepackage{afterpage}
\usepackage[]{gitinfo2}
% local
\usepackage{rubrics}
\usepackage{styling}
% Config
\pagestyle{empty}
% =========
% layout
\setulmarginsandblock{0.5in}{0.7in}{*}
\setlrmarginsandblock{0.5in}{0.5in}{*}
\checkandfixthelayout
% headers
\setsecnumdepth{chapter}
\setsecheadstyle{\LARGE\raggedright\centering}
\setsubsecheadstyle{\Large\scshape\raggedright\centering}
\setsubsubsecheadstyle{\color{red}\large\scshape\raggedright\centering}
\setparaheadstyle{\color{red}\large\raggedright\centering}
\setafterparaskip{1.5ex plus .2ex}
% multicols
\renewcommand{\columnseprulecolor}{\color{red}}
\setlength{\multicolsep}{0em}
\setlength{\columnseprule}{0.4pt}
\colseprulecolor{red}
% page
\openany

\makeatletter
\newcommand{\oneraggedpage}{\let\mytextbottom\@textbottom
  \let\mytexttop\@texttop
  \raggedbottom
  \afterpage{%
    \global\let\@textbottom\mytextbottom
    \global\let\@texttop\mytexttop}}
\makeatother
\renewcommand{\X}{\textcolor{red}{\grecross}}
\renewcommand{\r}{\textcolor{red}{\gothRbar.}}
\renewcommand{\v}{\textcolor{red}{\gothVbar.}}

\renewcommand{\red}[2][\centering]{%do the red....
  % call with empty optional argument for justified text; or
  % \raggedright.]
  % \filbreak
  {\textcolor{red}{\noindent#2}#1}\par\nopagebreak\medskip\nopagebreak}


\newcommand{\ps}[2][]{%typeset `psalmus'
  \paragraph{Psalmus #2#1}}


% \medskip\nopagebreak
% {\textcolor{red}{\noindent Psalmus #2#1}\centering\par\nopagebreak\vspace{-0.5\baselineskip}\nopagebreak} }

\newcommand{\black}[1]{%say the black
  \textcolor{black}{#1}}

\newcommand{\dropcap}[2]{%
  \lettrine{\textcolor{red}{#1}}{#2}}

\newcommand{\oremus}{%
  \paragraph{\black{Oremus.}\hfill Oratio}
}

\newcommand{\domvob}{%
  \par
  \noindent\v\ Dóminus vobiscum.\\
  \r\ Et cum spíritu tuo. \textcolor{red}{\emph{vel:}}

  \noindent\v\ Dómine, exáudi oratiónem meam.\\
  \r\ Et clamor meus ad te véniat.
}

% modified from https://tex.stackexchange.com/questions/163334/using-lettrine-or-equivalent-inside-verse-environment

\def\startverse#1\\#2\\{%
  \begin{minipage}{4in}%
    \firstline#1\relax%
    \def\verselineB{#2}%
    \if Q\versalletter\def\descstrut{\strut}\else\def\descstrut{}\fi%
    \def\Versal{\textcolor{red}{%
        \scalerel*{$\fontsize{28}{30}\selectfont\versalletter$}%
        {\def\stacktype{L}\stackon{T\descstrut}{T}}}}%
    \setbox0=\hbox{\Versal\,}%
    \def\leftoffset{-.5\wd0}%
    \hspace*{\wd0}\hspace{\leftoffset}%
    \verselineA\\%
    \hspace*{\wd0}\hspace{\leftoffset}%
    \llap{\smash{\box0}}%
    \hspace{0.5em}\verselineB\strut%
  \end{minipage}\\%
}

\def\firstline#1#2 #3\relax{\def\versalletter{#1}\def\verselineA{\textsc{#2} #3}}

\begin{document}
\selectlanguage{latin}

\setlength{\parskip}{\medskipamount}
\setlength{\parindent}{0em}

% \renewcommand{\gregorioscore}[1]{}

\section{Incipit}
\label{sec:incipit}

\gregorioscore{../common/jubedomine.gabc}

% \gregorioscore{../common/noctem.gabc}

\reading{Lectio brevis}{1 Pet 5:8--9}

\begin{multicols}{2}
  \dropcap{F}{ratres:} Sóbrii estóte, et vigiláte: quia adversárius
  vester diábolus tamquam leo rúgiens círcuit, quærens quem dévoret:
  cui resístite fortes in fide.  \v~Tu autem, Dómine, miserére
  nobis. \r~Deo grátias.
\end{multicols}

\begin{verse}
  \v~Adiutórium nóstrum~\X~in nómine Dómini.\\
  \r~Qui fecit cælum et terram.
\end{verse}

\red{Examen conscientiæ, vel \black{Pater noster} totum secreto.}

\begin{multicols}{2}
  \red{Hebdomadarius facit confessionem:}

  \dropcap{C}{onfíteor} Deo omnipoténti, beátæ Maríæ semper Vírgini,
  beáto Michaéli Archángelo, beáto Ioánni Baptístæ, sanctis Apóstolis
  Petro et Paulo, et ómnibus Sanctis,
  et vobis, fratres,
  quia peccávi nimis, cogitatióne, verbo et
  ópere: mea culpa, mea culpa, mea máxima culpa. Ideo precor beátam
  Maríam semper Vírginem, beátum Michaélem Archángelum, beátum Ioánnem
  Baptístam, sanctos Apóstolos Petrum et Paulum, et omnes Sanctos,
  et vos, fratres,
  oráre pro me ad Dóminum Deum nostrum.

  \red{Respondet chorus:}

  \dropcap{M}isereátur tui omnípotens Deus, et dimíssis peccátis tuis,
  perdúcat te ad vitam ætérnam. Amen.

  \red{Chorus facit confessionem:}

  \dropcap{C}{onfíteor} Deo omnipoténti, beátæ Maríæ semper Vírgini,
  beáto Michaéli Archángelo, beáto Ioánni Baptístæ, sanctis Apóstolis
  Petro et Paulo, ómnibus Sanctis,
  et tibi, pater,
  quia peccávi nimis, cogitatióne, verbo et
  ópere: mea culpa, mea culpa, mea máxima culpa. Ideo precor beátam
  Maríam semper Vírginem, beátum Michaélem Archángelum, beátum Ioánnem
  Baptístam, sanctos Apóstolos Petrum et Paulum, omnes Sanctos,
  et te, pater,
  oráre pro me ad Dóminum Deum nostrum.

  \newpage\red{Hebdomadarius dicit:}

  \dropcap{M}isereátur vestri omnípotens Deus, et dimíssis peccátis vestris,
  perdúcat vos ad vitam ætérnam. Amen.

  \dropcap{I}{ndulgéntiam},~\X~absolutiónem et remissiónem peccatórum
  nostrórum tríbuat nobis omnípotens et miséricors Dóminus. Amen.%
  \footnote{
    In recitatione \emph{a solo}, et quando non præest sacerdos, confessionem ab
    omnibus fit semel ac simul, verbis «\ et vobis fratres\ », «\ et vos
    fratres\ » omissis, et «\ nostri\ » ac «\ nostris\ »  pro «\ vestri\ » ac
    «\ vestris\ » in formula absolutionis.
  }
\end{multicols}

\begin{verse}
  \v~Convérte~\X~nos Deus, salutáris noster.\\
  \r~Et avérte iram tuam a nobis.
\end{verse}

\red{Tonus Communis:}

\gregorioscore{../common/deusinadiutorium_simple.gabc}

\red{Tonus Sollemnis:}

\gregorioscore{../common/deusinadiutorium_solemn.gabc}
% 

\section{Psalterium}
\label{sec:psalterium}

\setlength{\vindent}{\vgap}


\gregorioscore{../psalm-antiphons/sundayantiphoninit.gabc}

\let\tmp\subsection
\renewcommand{\subsection}{\clearpage\tmp}

\ps{4}
\gregorioscore{perannum/dom1.gabc}

\begin{verse}
{2.~}Miserére \textbf{me}i,~* et exáudi orati\textit{ó}\textit{nem} \textbf{me}am.\\
{3.~}Fílii hóminum, úsquequo gravi \textbf{cor}de?~* ut quid dilígitis vanitátem et quǽri\textit{tis} \textit{men}\textbf{dá}cium?\\
{4.~}Et scitóte quóniam mirificávit Dóminus sanctum \textbf{su}um:~* Dóminus exáudiet me cum clamáve\textit{ro} \textit{ad} \textbf{e}um.\\
{5.~}Irascímini, et nolíte peccáre:~† quæ dícitis in córdibus \textbf{ve}stris,~* in cubílibus vestris \textit{com}\textit{pun}\textbf{gí}mini.\\
{6.~}Sacrificáte sacrifícium justítiæ,~† et speráte in \textbf{Dó}mino.~* Multi dicunt: quis osténdit \textit{no}\textit{bis} \textbf{bo}na?\\
{7.~}Signátum est super nos lumen vultus tui, \textbf{Dó}mine:~* dedísti lætítiam in \textit{cor}\textit{de} \textbf{me}o.\\
{8.~}A fructu fruménti, vini et ólei \textbf{su}i~* mul\textit{ti}\textit{pli}\textbf{cá}ti sunt.\\
{9.~}In pace in i\textbf{dí}psum~* dórmiam et \textit{re}\textit{qui}\textbf{é}scam;\\
{10.~}Quóniam tu, Dómine, singuláriter \textbf{in} spe~* con\textit{sti}\textit{tu}\textbf{í}sti me.\\
{11.~}Glória Patri, et \textbf{Fí}lio,~* et Spirí\textit{tu}\textit{i} \textbf{San}cto.\\
{12.~}Sicut erat in princípio, et nunc, et \textbf{sem}per,~* et in sǽcula sæcu\textit{ló}\textit{rum}. \textbf{A}men.\\
\end{verse}
\ps{90}
\gregorioscore{perannum/dom2.gabc}

\begin{verse}
{2.~}Dicet Dómino: Suscéptor meus es tu, et refúgium \textbf{me}um:~* Deus meus sperá\textit{bo} \textit{in} \textbf{e}um.\\
{3.~}Quóniam ipse liberávit me de láqueo ve\textbf{nán}tium,~* et a \textit{ver}\textit{bo} \textbf{á}spero.\\
{4.~}Scápulis suis obumbrábit \textbf{ti}bi:~* et sub pennis e\textit{jus} \textit{spe}\textbf{rá}bis.\\
{5.~}Scuto circúmdabit te véritas \textbf{e}jus:~* non timébis a timó\textit{re} \textit{no}\textbf{ctúr}no.\\
{6.~}A sagítta volánte in die,~† a negótio perambulánte in \textbf{té}nebris:~* ab incúrsu et dæmónio me\textit{ri}\textit{di}\textbf{á}no.\\
{7.~}Cadent a látere tuo mille,~† et decem míllia a dextris \textbf{tu}is:~* ad te autem non ap\textit{pro}\textit{pin}\textbf{quá}bit.\\
{8.~}Verúmtamen óculis tuis conside\textbf{rá}bis:~* et retributiónem peccató\textit{rum} \textit{vi}\textbf{dé}bis.\\
{9.~}Quóniam tu es, Dómine, spes \textbf{me}a:~* Altíssimum posuísti refú\textit{gi}\textit{um} \textbf{tu}um.\\
{10.~}Non accédet ad te \textbf{ma}lum:~* et flagéllum non appropinquábit taberná\textit{cu}\textit{lo} \textbf{tu}o.\\
{11.~}Quóniam Angelis suis mandávit \textbf{de} te:~* ut custódiant te in ómnibus \textit{vi}\textit{is} \textbf{tu}is.\\
{12.~}In mánibus por\textbf{tá}bunt te:~* ne forte offéndas ad lápidem \textit{pe}\textit{dem} \textbf{tu}um.\\
{13.~}Super áspidem et basilíscum ambu\textbf{lá}bis:~* et conculcábis leónem \textit{et} \textit{dra}\textbf{có}nem.\\
{14.~}Quóniam in me sperávit, liberábo \textbf{e}um:~* prótegam eum quóniam cognóvit \textit{no}\textit{men} \textbf{me}um.\\
{15.~}Clamábit ad me, et ego exáudiam eum:~† cum ipso sum in tribulati\textbf{ó}ne:~* erípiam eum et glorifi\textit{cá}\textit{bo} \textbf{e}um.\\
{16.~}Longitúdine diérum replébo \textbf{e}um:~* et osténdam illi salu\textit{tá}\textit{re} \textbf{me}um.\\
{17.~}Glória Patri, et \textbf{Fí}lio,~* et Spirí\textit{tu}\textit{i} \textbf{San}cto.\\
{18.~}Sicut erat in princípio, et nunc, et \textbf{sem}per,~* et in sǽcula sæcu\textit{ló}\textit{rum}. \textbf{A}men.\\
\end{verse}
\ps{133}
\gregorioscore{perannum/dom3.gabc}

\begin{verse}
{2.~}Qui statis in domo \textbf{Dó}mini,~* in átriis domus \textit{De}\textit{i} \textbf{no}stri.\\
{3.~}In nóctibus extóllite manus vestras in \textbf{san}cta,~* et benedí\textit{ci}\textit{te} \textbf{Dó}minum.\\
{4.~}Benedícat te Dóminus ex \textbf{Si}on,~* qui fecit cæ\textit{lum} \textit{et} \textbf{ter}ram.\\
{5.~}Glória Patri, et \textbf{Fí}lio,~* et Spirí\textit{tu}\textit{i} \textbf{San}cto.\\
{6.~}Sicut erat in princípio, et nunc, et \textbf{sem}per,~* et in sǽcula sæcu\textit{ló}\textit{rum}. \textbf{A}men.\\
\end{verse}

\let\subsection\tmp



\section{Hymnus}
\label{sec:hymnus}


\gregorioscore{../hymns/telucis_feriis.gabc}

\begin{paracol}{2}
  \setlength{\parindent}{0em}
  \setlength{\parskip}{0.2\baselineskip}
  Procul recédant sómnia,\\
  Et nóctium phantásmata :\\
  Hostémque nostrum cómprime,\\
  Ne polluántur córpora.
  \switchcolumn

  Præsta Pater omnípotens,\\
  Per Jesum Christum Dóminum\\
  Qui tecum in perpétuum\\
  Regnat cum Sancto Spíritu.  \textbf{Amen.}
\end{paracol}


\section{Capitulum et Responsorium Breve}

\reading{Capitulum}{Jer 14:9}

\begin{multicols}{2}
  \dropcap{T}{u} autem in nobis es, Dómine, et nomen sanctum tuum
  invocátum est super nos: ne derelínquas nos, Dómine, Deus
  noster. \r~Deo grátias.
\end{multicols}

\gregorioscore{../common/inmanus_duringyear.gabc}
\gregorioscore{../common/custodi.gabc}

\section{Canticum}

\gregorioscore{../canticles/salva_nos.gabc}

\reading{Canticum Simeonis}{Luc 2:29--32}

\gregorioscore{Nunc}
\begin{verse}
  {2.~}\emph{Quia} vidérunt \textbf{ó}culi \textbf{me}i~* salu\textbf{tá}re \textbf{tu}um,\\
  {3.~}\textbf{Quod} pa\textbf{rá}sti~* ante fáciem ómnium \textbf{po}pu\textbf{ló}rum,\\
  {4.~}\emph{Lumen} ad revelati\textbf{ó}nem \textbf{Gén}\textbf{ti}um,~* et glóriam plebis \textbf{tu}æ \textbf{Is}raël.\\
  {5.~}\emph{Glóri}a \textbf{Pa}tri, et \textbf{Fí}\textbf{li}o,~* et Spi\textbf{rí}tui \textbf{San}cto.\\
  {6.~}\emph{Sicut} erat in princípio, et \textbf{nunc}, et \textbf{sem}per,~* et in sǽcula sæcu\textbf{ló}rum. \textbf{A}men.\\
\end{verse}


\section{Ad Conclusionem}


\begin{multicols}{2}
  \domvob
  \oremus
  \dropcap{V}{ísita,} quǽsumus, Dómine, habitatiónem istam, et
  omnes insídias inimíci ab ea lónge repélle:~\+ Ángeli tui sancti
  hábitent in ea, qui nos in pace custódiant;~\* et benedíctio tua sit
  super nos semper.  Per Dóminum nostrum Iesum Christum, Fílium
  tuum:~\+ qui tecum vivit et regnat in unitáte Spíritus Sancti
  Deus,~\* per ómnia sǽcula sæculórum. \r~Amen.

  \domvob

  \noindent\v\ Benedicámus Dómino.\\
  \r\ Deo grátias.

  \paragraph{Benedictio}
  Benedícat et custódiat nos omnípotens et miséricors
  Dóminus,~\X~Pater, et Fílius, et Spíritus Sanctus.
\end{multicols}

\section{Antiphona Finalis Beatæ Mariæ Virginis}

\gregorioscore{../marian-antiphons/salve-simple.gabc}

\v~Ora pro nobis, sancta Dei Génetrix.
\r~Ut digni efficiámur promissiónibus Christi.


\begin{minipage}{\textwidth}
  \oremus
  \begin{multicols}{2}
    Omnípotens sempitérne Deus, qui gloriósæ Vírginis Matris
    Maríæ corpus et ánimam, ut dignum Fílii tui habitáculum éffici
    mererétur, Spíritu Sancto cooperánte, præparásti: da, ut, cujus
    commemoratióne lætámur, ejus pia intercessióne, ab instántibus malis
    et a morte perpétua liberémur. Per eúndem Christum Dóminum
    nóstrum.\r~Amen.
  \end{multicols}
\end{minipage}

\subsection{Conclusio}

\begin{verse}
  \v~Divínum auxílium máneat semper nobíscum.\\
  \r~Amen.
\end{verse}

\begin{center}
  \+
\end{center}
\end{document}

%%% Local Variables:
%%% mode: latex
%%% TeX-engine: luatex
%%% End: